\part{Phase Three: Automate dependencies linking Dashboard with 3rd ticketing
tools (read/write). Tools like Trello, RT, sharepoint}
\chapter{Introduction}
\label{c_phasethree}

\section{}

\subsection{}

\subsubsection{}

\begin{figure}[ht!]
	\centering
   	\includegraphics[width=1\textwidth]{}
   	\caption{}
   	\label{f_}
\end{figure}

\begin{itemize}
  
  \item Capacidades del dispositivo
  
  Tenemos variables del tipo \textsf{Rboolean} que deberemos inicializar a su valor
  correcto cuando se inicializa el dispositivo por primera vez. Un ejemplo de
  estas capacidades es si el dispositivo es capaz o no de redimensionar el
  gráfico.
  
  \item Métodos para el tratamiento de gráficos
  
  Estos métodos son los que va a llamar el motor gráfico de R para dibujar en el
  dispositivo y saber el estado del mismo. Para ello el controlador deberá
  implementarlos.
  
  \item Posición del dispositivo
  
  Tenemos las variables left,top,right y bottom que el controlador se encargará
  de ir actualizando.
  
  \item Características que debe tener el dispositivo al iniciarse
  
  Estas características son que fuente va a tener cuando se cree el device, el
  color de fondo y el color de pintado entre otras cosas.
  
\end{itemize}