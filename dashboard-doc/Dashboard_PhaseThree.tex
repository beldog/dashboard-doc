\begin{part}{Conclusions and Future work}

%\chapter{Introduction}
%\label{c_phasethree}


\chapter{Conclusions}
There is a wide variety of ways to do the same thing, also frameworks and
technologies to use. The ones used until now on this project are just an example
of the technologies that fits the project needs but it doesn't mean they are the best
ones, but at least covers the main objective being compliance with it in a
short-mid term. But, as technology changes every day, the most important part is
to isolate (or decouple) your main business logic out of the technology used.
So, if in the future you are force to change the ecosystem, at least you can ensure your
logic will not change drastically and you can migrate it from one system to
another.\\

Building a tool usually requires significant time for coding what you have
defined on the Use/Business cases, but a major part of the time is also used to
assess and evaluate which tools/frameworks provide the open source community
that could be reused and fit on your own project. This investment on reusing
components from 3rd is a common practise and it will safe value time that you
could invest on other tasks. 
In the same way that reusing Bootstrap or Google
Charts provides a good result without any need to invest time on coding it from
scratch rather than read the API.\\

All in all, the tool that has been built on this time still has room for
improvements, but it is already in a stage where it can be used on a daily basis
to track daily project's events and in parallel keep working on extend it and
provide new features that will help the Project managers to track, analyze and
decide next actions for their projects.

\chapter{Future work}
We left out of this part improvements and things to do better, but they will be
covered on future work, such as:


\begin{itemize}
	\item Refactor database connectivity using a proper JPA based
	framework\footnote{Java Persistence API:
	\url{http://www.oracle.com/technetwork/articles/javaee/jpa-137156.html}}.
	\item Remove data-format coupling returned to the client.
	\item Refactor client making it usable for the final user (requires assessment
	and evaluation of main actions to carry out by the user and role).
	\item Refactor client code architecture using Backbone.js and Underscore.js
	\item Implement rest of listeners and actions required on CHAP Links Timeline
	chart to build a full interactive chart.
	\item Automate dependencies linking projects with 3rd ticketing tools.
	\item Define and implement a performance and evaluation system able to detect
	risk, and calculate&evaluate main KPIs.
\end{itemize}
\end{part}
